\documentclass[8pt,pdf,handout]{beamer}
\usepackage{pervasives}

\begin{document}
\mode<presentation>{}

\title{First-Class Continuations: What and Why}
\author{\large Arjun Guha}
\institute{\normalsize University of Massachusetts Amherst}
\date{}

\begin{frame}
\titlepage

\end{frame}

\begin{frame}

\frametitle{Plan}

\textbf{Today's topic}: An introduction to \emph{first-class continuations}

\vskip 0.5em

\textbf{Tomorrow's topic}: How to implement first-class continuations by source-to-source
translation

\pause

\vskip 3em

\textbf{Today's meta-topic}: The \emph{expressiveness} of programming language
features

\vskip 0.5em

\textbf{Tomorrow's meta-topic}: How to conduct research on a full-fledged
programming language, without losing your sanity

\vskip 3em

We will use basic JavaScript for all code samples. Even if you don't know the
language, it should be straightforward to follow. Please ask questions if you
have trouble with the syntax.

\vskip 3em

\pause

\begin{alertblock}{Please interrupt and ask questions!}
We are going to see a few code
examples that are very challenging to understand.

\vskip 0.5em

Ask questions of the form, \emph{``What if we change the example to do X
instead?''} We will change the example and run it to see how it behaves.
\end{alertblock}

\end{frame}

\begin{frame}
\frametitle{Control Operators}

\begin{definition}
A \emph{control operator} is a programming language construct that
changes the normal flow of execution in a program.
\end{definition}

First-class continuations are a kind of control operator.

\vskip 1em

Some other control operators that are more familiar:

\begin{itemize}

\item \emph{Conditionals}: \lstinline|if| and \lstinline|switch|

\item \emph{Loops}: \lstinline|for| and \lstinline|while|

\item \emph{Structured jumps}: \lstinline|break| and \lstinline|continue|

\item \emph{Exceptions}: \lstinline|try catch| and \lstinline|try finally|

\end{itemize}

\vskip 1em

\pause

Recent versions of JavaScript support some additional control
operators:

\begin{itemize}

\item \emph{Asynchronous functions:} \lstinline|async function|
and \lstinline|await|

\item \emph{Generator functions:} \lstinline|function*|, \lstinline|yield|,
and \lstinline|yield*|

\end{itemize}

\begin{alertblock}{}
None of this is JavaScript specific: you can find all the control operators
above in several other programming languages.
\end{alertblock}

\end{frame}

\begin{frame}

\frametitle{Why New Control Operators?}

\begin{itemize}

  \item The right control operator can make programs much easier to read
  and write. We will show how, using JavaScript's asynchronous functions
  and generator functions.

  \item First-class continuations make it possible to build several control
  operators as a library (i.e., without building them into the language).

\end{itemize}

\end{frame}

\begin{frame}[fragile]
\frametitle{Synchronous I/O}

\begin{definition}
A \emph{synchronous} or \emph{blocking} I/O operation blocks execution while
performing I/O and then resumes execution with the result of the I/O operation.
\end{definition}

\vskip 1em
For example, a function with the following type would synchronously
download an image:
\begin{lstlisting}
loadImage(url: string) => Image
\end{lstlisting}
We could then display the image:
\begin{lstlisting}
// drawImage(image: Image) => void
function drawImage(image) {
  document.body.appendChild(image);
}

let image = loadImage("example.jpg");
drawImage(image);
\end{lstlisting}

\vskip 1em

\pause

\begin{alertblock}{}
In JavaScript, (almost) all I/O is asynchronous.
It is impossible to write a synchronous
\lstinline|loadImage| function in JavaScript.
\end{alertblock}

\end{frame}

\begin{frame}[fragile]
\frametitle{Asynchronous I/O}

\begin{definition}
\emph{Asynchronous} or \emph{nonblocking} I/O performs the
operation in the background and applies
\emph{callback function} when the I/O result is available.
\end{definition}

\vskip 1em

We can implement an asynchronous function to load images:
\begin{lstlisting}
loadImage(url: string, callback: Image => void) => void
\end{lstlisting}

\pause

So, to load two images in sequence (\linkex{callbacks.html}):

\begin{lstlisting}
function callback2(image2) {
  drawImage(image);
}

function callback1(image1) {
  drawImage(image);
  loadImage("example2.jpg", callback2);
}

loadImage("example1.jpg", callback1);
\end{lstlisting}

\vskip 1em

\pause

\begin{alertblock}{}
We need a new callback function to wait for the
result of each new asynchronous operation.
\end{alertblock}

\end{frame}

\begin{frame}[fragile]

\frametitle{Async Functions}

\begin{definition}
An \emph{async function} can start an asynchronous task,
suspend its own execution, and then resume with the result of
the task when the task completes.
\end{definition}

(\linkex{async.html})
\begin{lstlisting}
function loadImage(url) {
  // implementation omitted. Turns callbacks into Promises.
}

async function F() {
  let image1 = await loadImage("example1.jpg");
  drawImage(image1);
  let image2 = await loadImage("example2.jpg");
  drawImage(image2);
}
\end{lstlisting}

\begin{itemize}

\item Within an async function, we use the \lstinline|await| keyword to
call another async function.

\item We can read the program from top to bottom (contrast to
the callback-based approach).

\end{itemize}

\end{frame}


\begin{frame}[fragile]
\frametitle{Generator Functions}

\begin{definition}
A \emph{generator function} is a special kind of function
that suspends its execution when it produces a value for
the caller. The caller may then resume the generator function 
to make it produce the nex value (if any).
\end{definition}

Ordinary functions \emph{do not} work this way: they
run to completion and cannot be suspended.

\vskip 1em

\begin{lstlisting}
function* makeThreeGen() {
  yield 1;
  yield 2;
  yield 3;
}
\end{lstlisting}

\vskip 1em

\begin{lstlisting}
let gen = makeThreeGen();
console.log(gen.next().value); // displays 1
console.log(gen.next().value); // displays 2
\end{lstlisting}


\end{frame}

\begin{frame}[fragile]
\frametitle{Unbounded Generators}

The following example helps illustrate that the \lstinline|yield|
statement truly suspends execution of the generator function:
\begin{lstlisting}
function* genNats() {
  let i = 0;
  while (true) {
    yield i;
    i = i + 1;
  }
}

let gen = genNats();

// Displays 0, 1, 2, 3, 4, 5, 6, 7, 8, 9
for (let i = 0; i < 10; i++) {
  console.log(gen.next().value);
}
\end{lstlisting}

If \lstinline|yield| did not suspend the generator function, the
loop in the generator would run forever.
\end{frame}

\begin{frame}

\frametitle{Some More Control Operators}

There are several other kinds of control operators that JavaScript
does not have:

\begin{itemize}

  \item \emph{Sampling executions from probability distributions}:
  \href{http://webppl.org}{WebPPL} implements a compiler from a small subset of
  JavaScript with extensions to support probabilistic programming.

  \item \emph{Backtracking search:} \href{http://tau-prolog.org}{Tau Prolog}
  implements a Prolog interpreter in JavaScript.

  \item \emph{(Green) Threads:} \href{https://webkit.org/blog/7846/concurrent-javascript-it-can-work/}{Concurrent JavaScript}

\end{itemize}

\pause

\begin{block}{Question}
Should JavaScript implement these natively too?
\end{block}

\pause

\vskip 1em

Using first-class continuations, we can implement each control operator above
as a small library. In essence, first-class continuations subsume a wide
variety of other control operators and language features.

\end{frame}

\begin{frame}
\frametitle{Semantics, Informally}

We need to understand three concepts before we get
to first-class continuations:

\begin{enumerate}

  \item Environments

  \item Active Expressions

  \item Continuations

\end{enumerate}

This will be an quick, informal introduction via examples. For a rigorous, formal
approach, see
\href{https://mitpress.mit.edu/books/semantics-engineering-plt-redex}{Semantics
Engineering with PLT Redex}.

\end{frame}

\begin{frame}[fragile]
\frametitle{Environments}

\begin{definition}
The \emph{environment} of a program maps variable names to values.
\end{definition}


\begin{columns}

\begin{column}{0.4\textwidth}
\begin{lstlisting}
let x = 100;
let y = 200;
console.log(x + y);
\end{lstlisting}
At the last line, the environment holds the values of
\lstinline|x| and \lstinline|y|.
\end{column}

\begin{column}{0.4\textwidth}
\begin{lstlisting}
function F(x) {
  console.log(x);
}

F(100);
F(200);
\end{lstlisting}
Each application evaluates the body of \lstinline|F| 
in a different environment.
\end{column}

\end{columns}

\end{frame}

\begin{frame}[fragile]
\frametitle{Active Expressions}

\begin{definition}
The \emph{active expression} in a program is the smallest part of the
program that the language will evaluate next.
\end{definition}

Examples:

\begin{columns}
\begin{column}{0.4\textwidth}
\begin{lstlisting}
console.log(2 + 3 * 4);
\end{lstlisting}
\begin{tikzpicture}[framed, remember picture,overlay,shift={(0,0)}]
\draw[solid,color=blue] (2.6,0.4) rectangle (3.7,0.85);
\end{tikzpicture}
\vskip -1em

\pause

\begin{lstlisting}
console.log(2 + 12);
\end{lstlisting}
\begin{tikzpicture}[framed, remember picture,overlay,shift={(0,0)}]
\draw[solid,color=blue] (1.9,0.4) rectangle (3.2,0.85);
\end{tikzpicture}
\vskip -1em

\pause

\begin{lstlisting}
console.log(14);
\end{lstlisting}
\begin{tikzpicture}[framed, remember picture,overlay,shift={(0,0)}]
\draw[solid,color=blue] (0,0.4) rectangle (2.8,0.85);
\end{tikzpicture}

\pause

Two things to note:

\begin{itemize}

\item A statement or the whole program can be the active expression.

\item Values cannot be active expressions.

\end{itemize}

\end{column}

\begin{column}{0.4\textwidth}
\begin{lstlisting}
if (5 > 10) {
  console.log("oops");
}
else {
  console.log("hi");
}
\end{lstlisting}
\begin{tikzpicture}[framed, remember picture,overlay,shift={(0,0)}]
\draw[solid,color=blue] (0.5,2.2) rectangle (2,1.9);
\end{tikzpicture}

\vskip -1em
\pause

\begin{lstlisting}
if (false) {
  console.log("oops");
}
else {
  console.log("hi");
}
\end{lstlisting}
\begin{tikzpicture}[framed, remember picture,overlay,shift={(0,0)}]
\draw[solid,color=blue] (0,0.3) rectangle (4.2,2.3);
\end{tikzpicture}

\vskip -1em
\pause

\begin{lstlisting}
console.log("hi");
\end{lstlisting}
\begin{tikzpicture}[framed, remember picture,overlay,shift={(0,0)}]
\draw[solid,color=blue] (0,0.4) rectangle (3.2,0.85);
\end{tikzpicture}
\end{column}

\end{columns}

\end{frame}

\begin{frame}[fragile]
\frametitle{Continuations}

\begin{definition}
The \emph{continuation} of an active expression is the rest of the program (i.e.,
excluding the active expression).
\end{definition}

Therefore, the continuation is an expression with a single ``hole'', where
the active expression used to be. We will write that hole as $\Box$.

\vskip 1em

For example:
\vskip 1em

\begin{tabular}{l|l|l}
Program & Active Expression & Continuation \\
\hline
\lstinline|console.log(2 + 3 * 4)| &
\lstinline|3 * 4| &
\pause \lstinline|console.log(2 + hole)| \\
\hline
\lstinline|console.log(2 + 12)| &
\lstinline|2 + 12| &
\pause \lstinline|console.log(hole)| \\
\hline
\lstinline|console.log(14)| &
\lstinline|console.log(14)| &
\pause \lstinline|hole| \\
\hline\end{tabular}

\pause
\begin{alertblock}{Note}
All programming languages have continuations. They are fundamental
for evaluation.
\end{alertblock}

To run a program, we:

\begin{enumerate}
  \item Identify the active expression and the continuation,
  \item Evaluate the active expression,
  \item Plug the result of evaluation into the hole that the active
  expression,
  \item Repeat, until the program does not have an active expression.
\end{enumerate}

\end{frame}

\begin{frame}[fragile]
\frametitle{Another Example of Continuations}

\scalebox{0.8}{
\begin{tabular}{|l|r|l|}
\hline \multirow{3}{*}{Step 1}
& Program & \lstinline|if (10 > 5) \{ log("o"  + "k"); \} else \{ log("oops"); \}| \\
\pause & Active Expression & \lstinline|10 > 5| \\
& Continuation & \lstinline|if (hole) \{ log("o"  + "k"); \} else \{ log("oops"); \}| \\
\hline \multirow{3}{*}{Step 2}
& Program & \lstinline|if (true) \{ log("o" + "k"); \} else \{ log("oops"); \}| \\
\pause & Active Expression & \lstinline|if (true) \{ log("o" + "k"); \} else \{ log("oops"); \}| \\
& Continuation & \lstinline|hole| \\
\hline \multirow{3}{*}{Step 3}
& Program & \lstinline|log("o" + "k");| \\
\pause & Active Expression & \lstinline|"o" + "k"| \\
& Continuation & \lstinline|log(hole);| \\
\hline \multirow{3}{*}{Step 4}
& Program & \lstinline|log("ok");| \\
\pause & Active Expression & \lstinline|log("ok");| \\
& Continuation & \lstinline|hole| \\
\hline
\end{tabular}
}

\end{frame}

\begin{frame}[fragile]

\frametitle{First-Class Continuations}

\begin{definition}
In programming languages that support \emph{first-class continuations},
it is possible to turn a continuation into a value.
\end{definition}

A continuation value stored in a variable:
\begin{lstlisting}
let k = (hole + 100);
\end{lstlisting}

\pause

An array of continuation values:
\begin{lstlisting}
let arr = [hole + 100, hole + 200];
\end{lstlisting}
A continuation passed as an argument to a function:
\begin{lstlisting}
f(hole + 100);
\end{lstlisting}

\vskip 1em

\pause

\begin{block}{Note}
The \lstinline|hole| notation is pseudocode. It is not possible to
directly write a continuation in a program, but it helps understand
what is happening under the hood. We will resolve this in a moment.
\end{block}

\end{frame}

\begin{frame}[fragile]
\frametitle{Applying a Continuation Value}

A continuation value can be applied to an argument. (Applying a continuation
value looks like applying a unary function.) When a continuation \lstinline|k|
is applied to a value \lstinline|v|:

\begin{enumerate}
  \item We discard the current continuation, and
  \item The continuation \lstinline|k| is restored and the hole
  gets filled with \lstinline|v|.
\end{enumerate}

\scalebox{0.8}{
\begin{tabular}{|l|r|l|}
\hline \multirow{4}{*}{Step 1}
& Program & \lstinline|let k = (hole + 1); log(100 + k(10))| \\
& Environment &  \\
& Active Expression & \lstinline|let k = (hole + 1)| \\
& Continuation & \lstinline|hole; log(100 + k(10))| \\
\hline \multirow{4}{*}{Step 2}
& Program & \lstinline|log(100 + k(10));| \\
& Environment & \lstinline|k = (hole + 1)| \\
& Active Expression & \lstinline|k(10)| \\
& Continuation & \lstinline|log(100 + hole);| \\
\hline \multirow{4}{*}{Step 3}
\pause
& Program & \lstinline|10 + 1;| \emph{This is \texttt{k} with \texttt{10} plugged in.} \\
& Environment & \lstinline|k = (hole + 1)| \\
& Active Expression & \lstinline|10 + 1| \\
& Continuation & \lstinline|hole| \emph{Continuation from Step 2 has been discarded} \\
\hline \multirow{2}{*}{Step 4}
& Result & \lstinline|11;| \\
& Environment & \lstinline|k = (hole + 1)| \\
\hline
\end{tabular}
}

\end{frame}

\begin{frame}

\frametitle{Note on \texttt{call/cc}}

\begin{alertblock}{Note}
If you have see continuations before, you may know about
\texttt{call/cc}. We are going to introduce an operator that is
similar, but not identical to \texttt{call/cc}.
\end{alertblock}

\end{frame}

\begin{frame}[fragile]

\frametitle{Capturing Continuations}

Languages with first-class continuations provide a primitive operator
that can turn its own continuation into a value (``capturing the current
continuation'').

\vskip 1em

We are going to introduce a new unary operator called \lstinline|control|,
which takes a unary function as an argument. When applied to a function
\lstinline|f|, \lstinline|control(f)|:

\begin{enumerate}

  \item Turns its own continuation into a continuation value \lstinline|k|, and
  \item Applies the function to the continuation value \lstinline|f(k)| in
  the empty continuation.
\end{enumerate}

\vskip 1em

The following example is fairly simple, since \lstinline|F| does not
use \lstinline|k| (\editorex{example1.js}):

\scalebox{0.7}{
\begin{tabular}{|l|l|l|}
\hline \multirow{4}{*}{Step 1}
& Program & \lstinline|function F(k) \{ log(200); \}; log(100 + control(F));| \\
& Environment & \\
& Active Expression & \lstinline|function F(k) \{ log(200); \}|  \\
& Continuation & \lstinline|hole; log(100 + control(F));| \\
\hline \multirow{4}{*}{Step 2}
& Program & \lstinline|log(100 + control(F));| \\
& Environment & \lstinline|function F(k) \{ log(200); \};| \\
& Active Expression & \lstinline|control(F)|  \\
& Continuation & \lstinline|log(100 + hole);| \\
\hline \multirow{4}{*}{Step 3}
\pause
& Program & \lstinline|log(200);| \\
& Environment & \lstinline|let k = log(100 + hole); function F(k) \{ log(200); \};| \\
& Active Expression & \lstinline|log(200);|  \\
& Continuation & \lstinline|hole| \\
\hline
\end{tabular}
}

\end{frame}

\begin{frame}[fragile]
\frametitle{Applying a Captured Continuation (Example 1)}

In the following example, the \lstinline|throw| statement is never reached
(\editorex{example2.js}):

\begin{lstlisting}
function F(k) {
  k(200);
  throw "bad";
}
log(100 + control(F));
\end{lstlisting}

\scalebox{0.8}{
\begin{tabular}{|l|r|l|}
\hline \multirow{5}{*}{Step 1}
& Program & \lstinline|function F(k) \{ k(200); throw "bad";  \}; log(100 + control(F));| \\
& Environment & \\
& Active Expression & \lstinline|function F(k) \{ k(200); throw "bad";  \};|  \\
& Continuation & \lstinline|hole; log(100 + control(F));| \\
\hline \multirow{5}{*}{Step 2}
& Program & \lstinline|log(100 + control(F));| \\
& Environment & \lstinline|function F(k) \{ k(200); throw "bad";  \};| \\
& Active Expression & \lstinline|control(F)|  \\
& Continuation & \lstinline|log(100 + hole);| \\
\hline \multirow{5}{*}{Step 3}
\pause
& Program & \lstinline|k(200); throw "bad"| \\
& Environment & \lstinline|let k = log(100 + hole);| \\
& Active Expression & \lstinline|k(200)|  \\
& Continuation & \lstinline|hole; throw "bad";| \\
\hline \multirow{5}{*}{Step 4}
\pause
& Program & \lstinline|log(100 + 200);| \\
& Environment & \\
& Active Expression & \lstinline|100 + 200|  \\
& Continuation & \lstinline|log(hole);| \\
\hline
\multicolumn{3}{c}{$\cdots$}
\end{tabular}
}

\end{frame}

\begin{frame}[fragile]
\frametitle{Applying Captured Continuations (Example 2)}

Trace the execution of this program:

\begin{lstlisting}
let i = 0;
let saved = "nothing";
function handler(k) {
  saved = k;
  saved("Start");
}

log(control(handler));
if (i < 3) {
  i = i + 1;
  saved(i);
}
\end{lstlisting}

\end{frame}

\begin{frame}[fragile]
\frametitle{Building New Control Operators with First-Class Continuations}

\begin{itemize}

  \item Code that uses first-class continuations directly is usually very hard
  to read.

  \item But, we can use first-class continuations to build other control
  operators that are more straightforward.

\end{itemize}

\end{frame}

\begin{frame}[fragile]
\frametitle{A Countdown Program}

\begin{block}{Programming Challenge}
Write a program that counts down seconds, displaying: ``Three'', ``Two'', ``One'', ``Liftoff!''.
\end{block}

\pause

Solution (\editorex{liftoff_callbacks.js}):
\begin{lstlisting}
function callback3() {
  console.log("Liftoff!");
}

function callback2() {
  console.log("One");
  setTimeout(callback3, 1000);
}

function callback1() {
  console.log("Two");
  setTimeout(callback2, 1000);
}

console.log("Three");
setTimeout(callback1, 1000);
\end{lstlisting}

\pause

\begin{alertblock}{WTF}
We cannot write a blocking sleep function in JavaScript (without busy-waiting).
\end{alertblock}

\end{frame}

\begin{frame}[fragile]
\frametitle{Application: The Sleep Operator}

Using first-class continuations, we can simulate a synchronous
\lstinline|sleep| function:
\begin{lstlisting}
function sleep(n) {
  function sleeper(k) {
    setTimeout(k, n);
  }
  control(sleeper);
}
\end{lstlisting}

\pause

The countdown program, refactored:
\begin{lstlisting}
console.log("Three");
sleep(1000);
console.log("Two");
sleep(1000);
console.log("One");
sleep(1000);
console.log("Liftoff!");
\end{lstlisting}

\vskip 1em

\pause

We can take any asynchronous function and simulate synchronous execution
following this recipe. This is effectively what \lstinline|async function| and
\lstinline|await| do.

\end{frame}

\begin{frame}[fragile]
\frametitle{Application: Cooperative Threads}

\begin{definition} \emph{Cooperative threads} run several logical threads on a
single physical thread, thus there is no parallelism and only one thread is
running at a time (other threads are suspended in background). Moreover, the
running thread has to explicitly yield control of the physical thread for
another thread to start running.
\end{definition}

\pause

Key ideas in the implementation:

\begin{enumerate}

  \item A global array of continuation values, where each continuation value is
  a suspended thread.

  \item To yield, we capture the continuation of the active thread, add it to
  the array, and apply one of the other continuations.

\end{enumerate}

\pause

\begin{columns}

\begin{column}{0.4\textwidth}
\begin{lstlisting}
let threads = [ ];

function createThread(f) {
  function threadFunc() {
    f();
    start();
  }
  threads.push(threadFunc);
  yieldThread();
}
\end{lstlisting}
\end{column}

\begin{column}{0.55\textwidth}
\begin{lstlisting}
function start() {
  if (threads.length > 0) {
    let nextThread = threads.shift();
    nextThread();
  }
}

function yieldThread() {
  function switcher(k) {
    threads.push(k);
    let kOther = threads.shift();
    kOther("resumed");
  };

  control(switcher);
}
\end{lstlisting}
\end{column}

\end{columns}

\end{frame}

\begin{frame}[fragile]
\frametitle{Using Cooperative Threads}

(\editorex{cooperative_threads.js})

\begin{lstlisting}
function threadA() {
  for (let i = 0; i < 10; i++) {
      yieldThread();
      console.log(i);
  }
})

function threadB() {
  for (let i = 100; i < 110; i++) {
      yieldThread();
      console.log(i);
  };
})

createThread(threadA);
createThread(threadB);
start(); // needed to activate other threads
\end{lstlisting}
\end{frame}

\begin{frame}
\frametitle{Conclusion}

\begin{itemize}

  \item I apologize if your head hurts

  \item Control operators can make certain kinds of programs much easier to
  write (e.g., generator functions and async functions)

  \item First-class continuations are a powerful primitive for building new
  control operators

  \item All examples online (including some that I have not covered)

  \item Tomorrow: How to implement continuations by source-to-source translation


\end{itemize}

\end{frame}
\end{document}
